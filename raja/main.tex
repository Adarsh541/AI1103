\documentclass{beamer}
\usepackage{listings}
\lstset{
%language=C,
frame=single, 
breaklines=true,
columns=fullflexible
}
\usepackage{subcaption}
\usepackage{url}
\usepackage{amsmath}

\usepackage{amsthm}

\usepackage{tikz}
\usepackage{graphicx}
\usepackage{tkz-euclide} % loads  TikZ and tkz-base
%\usetkzobj{all}
\usetikzlibrary{calc,math}
\usepackage{float}

\newcommand\norm[1]{\left\lVert#1\right\rVert}
\renewcommand{\vec}[1]{\mathbf{#1}}
\newcommand{\R}{\mathbb{R}}
\newcommand{\C}{\mathbb{C}}
\providecommand{\brak}[1]{\ensuremath{\left(#1\right)}}
\providecommand{\abs}[1]{\vert#1\vert}
\providecommand{\fourier}{\overset{\mathcal{F}}{ \rightleftharpoons}}
\newcommand{\myvec}[1]{\ensuremath{\begin{pmatrix}#1\end{pmatrix}}}
\providecommand{\mean}[1]{E[ #1 ]}
\providecommand{\sbrak}[1]{\ensuremath{{}\left[#1\right]}}
\providecommand{\cbrak}[1]{\ensuremath{\left\{#1\right\}}}
\usepackage[export]{adjustbox}
\usepackage[utf8]{inputenc}
\usepackage{amsmath}
\usetheme{Boadilla}
\title{My Presentation}
\author{Raja Ravi Kiran Reddy}
\institute{IITH(AI)}
\date{\today}
\begin{document}

\begin{frame}
\titlepage
\end{frame}
\begin{frame}{}
   \begin{align}
    \boldsymbol{\Sigma^{-1}}=\frac{4\pi}{\lambda}\begin{pmatrix}
\chi_1 & 0      & \cdots & 0       & 0 \\
0      & \chi_2 & \cdots & 0       & 0 \\
\vdots & \vdots & \ddots & \vdots  & \vdots  \\
0      & 0      & \cdots & \chi_{N}& 0\\
0      &0       &\cdots  &0        & 1
\end{pmatrix}
\end{align} 
\end{frame}
\begin{frame}{}
\begin{align}
    \boldsymbol{u}=[R_1,R_2,...,R_N,\bar{A_e}] \in \mathbb{R}^{N+1}
\end{align}
\end{frame}
\begin{frame}{}
  \begin{align}
      \mathit{p}(\boldsymbol{u}|dM_1,....,dM_N) \propto \mathit{p}(dM_1,....,dM_N|\boldsymbol{u})\mathit{p}(\boldsymbol{u})
  \end{align}  
\end{frame}
\begin{frame}{}
    \begin{align}
        p(\boldsymbol{u})\propto exp\brak{-\frac{1}{2}\boldsymbol{u}\boldsymbol{\Sigma^{-1}}\boldsymbol{u}^{\top}}
    \end{align}
\end{frame}
\begin{frame}{}
    \begin{align}
        \chi_i = \frac{1}{\binom{V_i}{2}}\sum_{k,j,k\neq j}^{\binom{V_i}{2}}\brak{dR_{kj_i}-\frac{\lambda}{4\pi}d\phi_{kj_i}}^2
    \end{align}
\end{frame}
\begin{frame}{}
    \begin{align}
       \mathit{p}(dM_1,....,dM_N|\boldsymbol{u})\propto exp\brak{-\frac{\sum_{i=1}^{N}\brak{dM_i-\hat{m}_i(\boldsymbol{u})}^2}{2\sigma^2}} 
    \end{align}
\end{frame}
\begin{frame}{}
    \begin{align}
        \mathcal{L}(\boldsymbol{u})\propto -\frac{\sum_{i=1}^{N}\brak{dM_i-\hat{m}_i(\boldsymbol{u})}^2}{2\sigma^2} + \boldsymbol{u}\Sigma^{-1}\boldsymbol{u}^{\top} 
    \end{align}
\end{frame}
\begin{frame}{}
    \begin{align}
        \hat{m}(\boldsymbol{u})=\hat{m}(\boldsymbol{u_0})+\boldsymbol{D}(\boldsymbol{u}-\boldsymbol{u_0})
    \end{align}
\end{frame}
\begin{frame}{}
    \begin{align}
        \boldsymbol{D} = \begin{pmatrix}
-\frac{\sqrt{\bar{A_e}}}{R_1^2} & \frac{\sqrt{\bar{A_e}}}{R_1R_2}      & \cdots & \frac{\sqrt{\bar{A_e}}}{R_1R_N}       & \frac{1}{2R_1\sqrt{\bar{A_e}}} \\
\frac{\sqrt{\bar{A_e}}}{R_2R_1}     & -\frac{\sqrt{\bar{A_e}}}{R_2^2} & \cdots & \frac{\sqrt{\bar{A_e}}}{R_2R_N}      & \frac{1}{2R_2\sqrt{\bar{A_e}}}\\
\vdots & \vdots & \ddots & \vdots  & \vdots  \\
 \frac{\sqrt{\bar{A_e}}}{R_NR_1}    & \frac{\sqrt{\bar{A_e}}}{R_NR_2}     & \cdots & -\frac{\sqrt{\bar{A_e}}}{R_N^2}&\frac{1}{2R_N\sqrt{\bar{A_e}}}  
\end{pmatrix}
    \end{align}
\end{frame}
\begin{frame}{}
\begin{align}
   \mathcal{L}^{\prime}&=\frac{1}{2}\boldsymbol{x}\boldsymbol{A}\boldsymbol{x}^{\top}-\boldsymbol{b}\boldsymbol{x}\\
   \boldsymbol{x}&= \boldsymbol{u}-\boldsymbol{u_0}\\
   \boldsymbol{A}&= \Sigma^{-1}-\frac{\boldsymbol{D}\boldsymbol{D}^{\top}}{\sigma^2}\\
   \boldsymbol{b}&=\boldsymbol{D}\frac{\brak{m(\boldsymbol{u})-\hat{m}(\boldsymbol{u})}}{\sigma^2}+\Sigma^{-1}\boldsymbol{u_0}
   \end{align}
\end{frame}
\end{document}